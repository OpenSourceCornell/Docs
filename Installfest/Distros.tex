\documentclass[11pt, letterpaper, oneside]{article}

\usepackage{../internal}

\title{Linux Distributions}
\author{Cornell Open Source Aficionados}
\maketitle

\section{Linux Mint}
Linux Mint was originally created to add a set of user interface improvements to Ubuntu. As such, virtually everything \emph{under the hood} is the similar to and compatible with Ubuntu. However, the Linux Mint interface has remained conservative even while Ubuntu has implemented its radically new Unity desktop environment. The main downside to choosing Linux Mint is that it is mostly like Ubuntu, but is not quite as mainstream, so you may run into a few quirks that are not as easy to resolve. This may be worth it if you prefer the more traditional interface and the additional tweaks that Linux Mint implements.

\section{Ubuntu}
Ubuntu has become the go-to Linux distribution for novice and experienced users alike. This is because the distribution strikes a good balance between new and stable software. Despite the second place ranking on \texttt{Distrowatch.com}, Ubuntu seems to have the largest user base of any Linux distribution. This wide user base means that virtually any issue you encounter has been encountered by someone else already, and a solution likely exists in the forums. The loudest gripes about Ubuntu are regarding the default Unity interface. Depending on whether or not you like Unity, this may be a sticking point for you too. Additionally, 

\section{Fedora}

\section{Debian}

\section{OpenSUSE}

\section{Arch Linux}


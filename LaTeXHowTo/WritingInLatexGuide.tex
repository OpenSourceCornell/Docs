\documentclass[10pt,letterpaper,oneside]{article}

\usepackage{../internal}

\newcommand{\topic}{Writing in \LaTeX~Guide}

\begin{document}
\section{Basic Document}
A very basic document would consist of something like the following in a plain text file with a \verb+.tex+ extension.
\begin{verbatim}
\documentclass{article}
\begin{document}
\end{document}
\end{verbatim}

\section{Text Content}
Text content should be composed just as you would write text in any other document---nothing special---except for a few tricks to do things you can't with plain text.
\begin{itemize}
\item Quotes are written as  \`{}\`{}quote here' '---with two back--ticks, and two apostrophes. For single quotes, use one back-tick and one apostrophe.
\item Hyphens should be made with one minus signs (ie~\verb+-+~), dashes indicating a range of numbers should be made with two minus signs (ie~\verb+--+~), and dashes in sentences should be made with three minus signs (ie~\verb+---+~).
\item Percent signs denote comments in \LaTeX~(which should be used) but to create a percent sign use \verb+\%+.
\item Dollar signs and ampersands should be prefixed with a backslash. (ie \verb+\$+ and \verb+\&+ )
\item Paragraphs are made by leaving a blank line, with no indentation necessary:
\begin{verbatim}
Paragraph number one text...

Paragraph number two text...
\end{verbatim}
\end{itemize}

\section{Sectioning}
A document can be separated into a hierarchical structure (and the table of contents automatically updated) by using the following lines in the document:
\begin{enumerate}
\item \verb+\section{Section Name}+
\item \verb+\subsection{SubSection Name}+
\item \verb+\subsubsection{SubSubSection Name}+
\item \verb+\paragraph{Paragraph Name}+
\item \verb+\subparagraph{SubParagraph Name}+
\end{enumerate}

\section{Tables}

\subsection{Including Tables}
For clarity and organization, I recommend that tables be composed as separate files and included in documents as follows.
\begin{verbatim}
\begin{table}[tb]
\input{tables/tablename.tex}
\caption{Caption for the Table}
\label{tab:referencetag}
\end{table}
\end{verbatim}
This example assumes that the table file is located in a sub-folder called \verb+tables+. Also, in this example, \verb+tab:referencetag+ is the text string which is used to reference the table in the text of the document, so make is descriptive (but without spaces or numbers).

\subsection{Making Tables}
Making tables in \LaTeX~can become very complex, so only what is necessary to make a simple table will be addressed here. (For further information, please visit the tables section [\verb+https://en.wikibooks.org/wiki/LaTeX/Tables+] of the \LaTeX~documentation website.) Figure~\ref{fig:sampletab} is an example of how to create a basic table in \LaTeX.

\begin{figure}[htb]
\begin{minipage}[c]{0.5\linewidth}
\begin{verbatim} 
\begin{tabular}{ r || r | c | l }
Parameter & Num1 & Num2 & Num3 \\ \hline \hline
Param1 & 23 & 32 & 86 \\
Param2 & 26 & 51 & 19 \\ \hline
Param3 & 32 & 80 & 31 \\ 
\end{tabular}
\end{verbatim}
\end{minipage}
\hspace{.5cm}
\begin{minipage}[c]{0.5\linewidth}
\begin{tabular}{ r || r | c | l }
Parameter & Num1 & Num2 & Num3 \\ \hline \hline
Param1 & 23 & 32 & 86 \\
Param2 & 26 & 51 & 19 \\ \hline
Param3 & 32 & 80 & 31 \\ 
\end{tabular}
\end{minipage}
\caption{Sample Table}
\label{fig:sampletab}
\end{figure}

As shown, ampersand symbols (~\verb+&+~) are used to separate columns within a row. Rows are placed on new lines with two backslashes (~\verb+\\+~) separating them. To define the alignment of each column, \verb+r c l+ are used to denote right, center, and left alignment, respectively. Pipes \verb+|+ are used to denote either one or two lines dividing columns. To divide rows, \verb+\hline+ is placed in between the rows it will separate.

\section{Figures}
Figures (images) are included in a document in much the same way as tables.
\begin{verbatim}
\begin{figure}[tb]
\includegraphics{figures/imagename.png}
\caption{Caption for the Image}
\label{fig:referencetag}
\end{figure}
\end{verbatim}
This example assumes that the image file is located in a sub--folder called \verb+figures+. It should be noted that \LaTeX~accepts PNG and JPEG images. For simplicity, do not include spaces in the filename. Also, in this example, \verb+fig:referencetag+ is the text string which you will use to reference the figure in the text of the document, so make is descriptive (but without spaces or numbers).

Note: \verb+\usepackage{graphicx}+ must be before \verb+\begin{document}+ if you would like to include images. 

\section{Internal Referencing}
In a technical document, one is often required to reference tables, figures and other sections. To reference tables or figures use \verb+Table~\ref{tab:referencetag}+ and \verb+Figure~\ref{fig:referencetag}+ respectively.

To reference a section, one must first include a label to reference the section with. Under the \verb+\section{. . .}+ line, \verb+\label{sec:sectiontag}+ can be used to label a section where \verb+sectiontag+ is a text string related to the content. To then reference this section from another, use \verb+Section~\ref{sec:sectiontag}+.

\section{Equations}
A complete explanation of formatting equations in \LaTeX is far beyond this document, though Figure~\ref{fig:eq} gives an example. Further information can be found in the mathematics section [\verb+https://en.wikibooks.org/wiki/LaTeX/Mathematics+] of the \LaTeX~documentation website.

\begin{figure}[htb]
\begin{minipage}[c]{0.5\linewidth}
\begin{verbatim} 
\begin{equation}
x=\sin \left( \cfrac{1}{\sqrt[3]{\omega_0}} \right)
\end{equation}
\end{verbatim}
\end{minipage}
\hspace{.5cm}
\begin{minipage}[c]{0.5\linewidth}
\begin{equation}
x=\sin \left( \cfrac{n^{5}}{\sqrt[3]{\omega_0}} \right)
\end{equation}
\end{minipage}
\caption{Sample Equation}
\label{fig:eq}
\end{figure}

\section{Packages}
If any special features of \LaTeX~are required to typeset the desired content, they will likely require additional packages. To include a package, simply put \verb+usepackage{packagename}+ between the \verb+\documentclass+ and \verb+\begin{document}+ lines.

\section{Pronunciation}
\LaTeX~ is generally pronounced either LAY-TEK or LAH-TEK (though I prefer the latter due to the fact that it was derived from Leslie Lamport's last name) with the emphasis on the second syllable.

\section{Further Information}
All the functions of \LaTeX~cannot be covered in one short document---this was just meant to get you going. The \LaTeX~documentation website [\verb+https://en.wikibooks.org/wiki/LaTeX+] is a great reference for all the finer details.

\section{Sample Document}
\begin{verbatim}
\documentclass[10pt,letterpaper,oneside]{article}

\usepackage{fullpage}

\begin{document}
\author{Your Name}
\title{The Title}
\maketitle

\section{Some Section}
Content!

\end{document}
\end{verbatim}

\section{Software}
To get started with \LaTeX~all you really need is a text editor and a \LaTeX~\emph{distribution}, however I recommend installing a front end if you prefer not to use the command line---or just to get started even if you do.

On all operating systems, I recommend installing the TeXworks front-end for \LaTeX~ which has syntax highlighting and will show you your finished document alongside the plain text file.
\subsection{GNU/Linux}
TeX Live is the most common \LaTeX~\emph{distribution} used on GNU/Linux, and is likely available through your package manager (ie. \verb+sudo apt-get install texlive+)
\subsection{Mac OS}
MacTeX is the most common \LaTeX~\emph{distribution} used on Mac OS, and can be installed the same as any other piece of software.
\subsection{Microsoft Windows}
MiKTeX is the most common \LaTeX~\emph{distribution} used on Microsoft Windows, and can be installed the same as any other piece of software.

\end{document}

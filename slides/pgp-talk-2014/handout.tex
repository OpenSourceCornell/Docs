\documentclass{article}
\usepackage{url}
\usepackage[margin=0.75in]{geometry}
\usepackage{listings}
\pagestyle{empty}

\begin{document}
\part*{Getting Started with GPG on OS X}
The GPG Suite for OS X gives you a simple GUI for creating and
managing your GPG keys, as well as plugins for integrating PGP with
your OS X programs.  The following instructions take you through
installing GPG Suite and generating a keypair.  The settings suggested
here will increase the security of your key and are strongly
recommended!

If in doubt, come on \texttt{\#opensourcecornell} on Freenode and ask
for help!

\section*{Install GPG Suite and Setup Keypair}

\begin{enumerate}
\item Go to \url{https://gpgtools.org/} and download the \textit{GPG
    Suite}.
\item Run the installer.
\item The application ``GPG Keychain Access'' will pop up after the
  installation is complete.  This will allow you to generate a new
  keypair.  Keep ``Upload key after generation.'' unchecked.  Enter
  your name and email address, click ``Advanced Options,'' and
  increase the key length to 4096.  You don't need (or want) a
  comment.  \textbf{Take note of when your key expires!  You can
    always change the expiration date later!}
\item Generate the keypair.  You'll need to enter a passphrase and
  wait for the keypair to be generated.  This passphrase is the last
  line of defense if your private key is compromised, and you will
  need to enter it when you sign or decrypt messages.  \textbf{Do not
    forget this passphrase!}
\item Once your keypair has been generated, select it, go to the menu,
  and select ``Key$>$Generate Revoke Certificate\dots''.  Don't give a
  reason (this is in case anything ever happens).  Your revocation
  certificate will let you revoke this key if it ever gets stolen.
  Keep it in a secure place (not, for instance, on your desktop).
\item Go to the menu and select ``GPG Keychain
  Access$>$Preferences\dots''.  Change the keyserver to
  \url{hkps://hkps.pool.sks-keyservers.net}.  The HKPS protocol is
  more secure, because it sends the keys over an encrypted
  connection.
\item If you have more email addresses and names that you plan to use
  this key under, double click on the key, go to the User IDs tab, and
  click the first \texttt{+} button.  You will be prompted to add
  another name and email address (and comment, which again, you
  probably don't need or want).  By selecting a User ID, you can click
  the ``primary'' button to make it the name and email that others
  will see by default.
\item When you are done setting up your key, select it, go to the
  menu and select ``Key$>$Send public key to Keyserver''.
\end{enumerate}

\section*{Integration with Finder}
You can now sign, encrypt, and decrypt files in Finder using your new
key.

\begin{enumerate}
\item Open the directory containing the file you want to sign,
  encrypt, or decrypt.
\item Select the file in Finder.
\item\textbf{To sign the file:} Go to ``Finder$>$Services$>$OpenPGP:
  Sign File''.
\item\textbf{To encrypt the file:} Go to
  ``Finder$>$Services$>$OpenPGP: Encrypt File''.
\item\textbf{To decrypt the file:} Go to
  ``Finder$>$Services$>$OpenPGP: Decrypt File''.
\end{enumerate}

\section*{Integration with Apple Mail}
You should be able to sign messages with your new key, and encrypt
messages to anyone whose public key you have.

\begin{enumerate}
\item Start composing a message.  You'll see two extra buttons in the
  window: a lock and a star.
\item\textbf{To sign your message:} (a good idea for all your
  messages) make sure there is a check in the star button by clicking
  it.
\item\textbf{To encrypt your message:} (if you have the recipient's
  public key) make sure the lock is closed by clicking it.
\end{enumerate}

You will also be able to receive and decrypt encrypted messages to
your key, and verify signatures of signed messages if you have the
sender's public key.  To do this, just open the message from your
inbox.

\section*{Signing Someone Else's Key}
When you sign someone else's key, you are saying that you have
verified that the information in the key is correct and that they own
the key.  To do this, you will need to see photo identification.
\textbf{Never, never sign a key if you haven't verified all of its
  information!}

Next, generally after the keysigning party, you do the actually
keysigning.  You need to make sure that the person owns all the email
addresses they say they do.  The way you verify this is a little
convoluted: for each User ID, you sign it separately, export it, and
send them encrypted to the email address you signed.  If the key has
three User IDs, and the person only controls two of them, only those
two emails can be encrypted, and so you will effectively only have
signed those two!

Unfortunately, the GPG Suite doesn't give a good way to do this.  You
will need to do the following:

\begin{enumerate}
\item Once you have verified their name, download their public key by
  going to ``Key$>$Retrieve from Keyserver\dots''.
\item Export the original key by selecting it and clicking ``export''.
\item Double click the key and go to the ``User IDs'' tab.
\item For each User ID, do the following exactly:
  \begin{enumerate}
  \item Select the User ID.
  \item Click the \texttt{+} button below the ``Signatures'' field.
  \item Select how carefully you checked this key.  This should almost
    always be either ``I will not answer'' (which is perfectly fine)
    or ``I have done casual checking.''
  \item You may uncheck the box that says ``Signature expires'' if you
    want.  Most signatures do not expire.  It is slightly more secure
    if your signature expires, but you will need to keep track of this
    and go to another keysigning party with the person and repeat this
    entire section again when resigning.
  \item After generating the signature, close the ``Key Inspector''
    window and export the new key as a separate file.
  \item Encrypt this file, and send it by email to the email address
    you just signed.
  \item Delete the key from ``GPG Keychain Access'' by going to
    ``Key$>$Delete,'' and import it again from the original file.  If
    you do not do this, the next exported key will have more than one
    User ID signed, which defeats the purpose of this!  Go back to the
    ``Key Inspector'' window as above.
  \end{enumerate}
\item When the owner receives all the signatures, they will upload the
  new key to a keyserver, which you can then download.
\end{enumerate}

\section*{Having Your Own Key Signed}
If you are on the receiving end of the above process, be sure you
bring a photo ID (or more!) and your key information with you to the
keysigning.  Be sure to have written copies of your key id,
fingerprint, and all your user IDs to give to the keysigner, so they
can sign your key after the party.  You can find this by
double-clicking on your key.

Following the party, check your email addresses for encrypted messages
with the signatures.  Decrypt the messages, and import the attachments
into ``GPG Keychain Access''.  This will apply the signatures to your
key.  Then, select the key and go to ``Key$>$Send public key to
Keyserver'' so others can download the newly signed key.

\newpage
\part*{Getting Started with GPG on GNU/Linux or UNIX}
Most GNU or UNIX systems come with GPG, either installed by default or
in their package repositories.  This page assumes that you have
installed GPG.  (As a hint, look for a package like \texttt{gpg},
\texttt{gnupg}, or (in the case of Debian-derived systems)
\texttt{gpg2}.)

Most of the following are commands to be entered in the terminal.  (As
a note, Debian-derived systems should use the \texttt{gpg2} command in
place of the \texttt{gpg} command.)

If in doubt, come on \texttt{\#opensourcecornell} on Freenode and ask
for help!

\section*{Setup GPG}
Although the default settings for GPG are mostly acceptible, it is
recommended that you change some of the settings for better security.
To do so, open up the file \url{~/.gnupg/gpg.conf} in your favorite
editor and do the following:

\begin{enumerate}
\item Search for a line that looks like \texttt{\#keyserver
    hkp://keys.gnupg.net}.  Comment all such lines out, and add the
  following line below it:

  \begin{lstlisting}
    keyserver hkps://hkps.pool.sks-keyservers.net
  \end{lstlisting}
\item At the end of the file, add the following lines:

  \begin{lstlisting}
    personal-cipher-preferences AES256 TWOFISH AES192 AES
    personal-digest-preferences SHA512 SHA384 SHA256
    personal-compress-preferences ZLIB BZIP2 ZIP
    cert-digest-algo SHA256
  \end{lstlisting}
\end{enumerate}
\section*{Setup Keypair}

\begin{enumerate}
\item Open a terminal.
\item Run the command \texttt{gpg -{}-gen-key}
\item When prompted, select option 1, ``RSA and RSA (default)''.
\item Use 4096 bits instead of the default.
\item Give an expiration date for the key (2y or 3y is a good choice).
  This can always be changed later, but it's good if you lose the key
  at any point in the future.  \textbf{Take note of when your key
    expires!  You can always change the expiration date later!}
\item Enter your name, enter your email address, but do not enter a
  comment.
\item Enter a passphrase and wait for the key to be generated.  This
  passphrase is the last line of defense if your private key is
  compromised, and you will need to enter it when you sign or decrypt
  messages.  \textbf{Do not forget this passphrase!}
\item Once your keypair has been generated, take note of the key ID
  that was printed to the terminal (it will look like \texttt{pub
    4096R/KEYID}).  Next, run the command \texttt{gpg -o revcert.asc
    -{}-gen-revoke \{KEYID\}}.  This will generate a revocation
  certificate, which allows you revoke this key if it ever gets
  stolen.  Don't give a reason (this is in case anything ever
  happens).  Keep it in a secure place (not, for instance, right in
  your home directory).
\item If you have more email addresses and names that you plan to use
  this key under, run the command \texttt{gpg -{}-edit-key \{KEYID\}}.
  At the prompt that appears, type \texttt{adduid}.  Again, enter your
  name and email (and no comment).  Do this for each email you want to
  add.  When you are done, select the primary user ID by running the
  commands \texttt{uid \{UID NUMBER\}} and \texttt{primary}.  Exit by
  typing \texttt{quit}, and be sure to save your changes.
\item When you are done setting up your key, run the command
  \texttt{gpg -{}-send-key \{KEYID\}} to submit your public key to the
  keyserver we set up.
\end{enumerate}

\section*{Signing, Verifying, Encrypting, and Decrypting Files}
You can now use the \texttt{gpg} command to sign, verify, encrypt, and
decrypt files:

\begin{lstlisting}
$ gpg -s filename             # Sign a file, makes filename.gpg
$ gpg --clearsign -a filename # Sign file in same file, makes filename.asc
$ gpg --verify filename.asc   # Verify a signature or clearsigned message
$ gpg -e filename             # Encrypt a file, makes filename.gpg
$ gpg -d filename.gpg         # Decrypt a file and print its contents
\end{lstlisting}

\section*{Integration with Thunderbird, Evolution, etc.}
Thunderbird, Mutt, Evolution, and KMail all integrate with GPG very
easily, but there isn't enough space here to describe how.  Either
Google it (Thunderbird uses something called Enigmail, the others do
so natively) or ask one of us to help.

\section*{Signing Someone Else's Key}
When you sign someone else's key, you are saying that you have
verified that the information in the key is correct and that they own
the key.  To do this, you will need to see photo identification.
\textbf{Never, never sign a key if you haven't verified all of its
  information!}

Next, generally after the keysigning party, you do the actually
keysigning.  You need to make sure that the person owns all the email
addresses they say they do.  The way you verify this is a little
convoluted: for each User ID, you sign it separately, export it, and
send them encrypted to the email address you signed.  If the key has
three User IDs, and the person only controls two of them, only those
two emails can be encrypted, and so you will effectively only have
signed those two!

Follow these steps:

\begin{enumerate}
\item Once you have verified their name, download their public key
  running \texttt{gpg -{}-recv-keys \{THEIRKEYID\}}
\item Export the original key by running \texttt{gpg -{}-armor -o
    key.orig -{}-export \{THEIRKEYID\}}
\item For each User ID, do the following exactly:
  \begin{enumerate}
  \item Run \texttt{gpg -{}-edit-key \{THEIRKEYID\}}, and at the
    prompt, type \texttt{uid \{UIDNUMBERTOSIGN\}}.
  \item Type \texttt{sign} to sign the user ID, and then \texttt{quit}
    to exit this prompt.
  \item Export this key to a different file: \texttt{gpg -{}-armor -o
      key.uid\textit{n}signed -{}-export \{THEIRKEYID\}}.
  \item Encrypt this file, and send it by email to the email address
    you just signed.
  \item Delete the signed key (\texttt{gpg -{}-delete-keys
      \{THEIRKEYID\}}) and reimport the original (\texttt{gpg
      -{}-import key.orig}).  If you don't do this, the next exported
    key will have more than one User ID signed, which defeats the
    purpose of this!  Repeat for the remaining User IDs.
  \end{enumerate}
\item When the owner receives all the signatures, they will upload the
  new key to a keyserver, which you can then download.
\end{enumerate}

\section*{Having Your Own Key Signed}
If you are on the receiving end of the above process, be sure you
bring a photo ID (or more!) and your key information with you to the
keysigning.  Be sure to have written copies of your key id,
fingerprint, and all your user IDs to give to the keysigner, so they
can sign your key after the party.

Following the party, check your email addresses for encrypted messages
with the signatures.  Decrypt the messages, and import the attachments
using \texttt{gpg -{}-import filename}.  This will apply the
signatures to your key.  Then, send the newly signed key to a
keyserver so others can download it by running the command \texttt{gpg
  -{}-send-keys \{KEYID\}}.

\newpage
\part*{Getting Started with GPG on Windows}
To use GPG on Windows, you'll be using the GPG4Win distribution.  Most
of the following are commands to be entered in the command prompt.
The full distribution of GPG4Win has a GUI application to generate
keys, but it is not reliable and is rather buggy, so we suggest you
just use the command line version for generating keys.

If in doubt, come on \texttt{\#opensourcecornell} on Freenode and ask
for help!

\section*{Install and Setup GPG4Win}
\begin{enumerate}
\item In your browser, navigate to \url{http://www.gpg4win.org/} and
  click on the download link.  You can download GPG4Win Lite, which
  does not contain the graphical key manager program (which is what we
  recommend, because it is very buggy on Windows) or the full GPG4Win
  version.  Either way, you will be creating your key from the command
  line, to ensure the key is made properly.
\item Follow the steps through the installation, and use the default
  steps.
\end{enumerate}

Although the default settings for GPG are mostly acceptible, it is
recommended that you change some of the settings for better security.
To do so, open up the file
\url{C:\Users\{User}\AppData\Roaming\gnupg\gpg.conf} in
your favorite editor and do the following:

\begin{enumerate}
\item Search for a line that looks like \texttt{\#keyserver
    hkp://keys.gnupg.net}.  Comment all such lines out, and add the
  following line below it:

  \begin{lstlisting}
    keyserver hkps://hkps.pool.sks-keyservers.net
  \end{lstlisting}
\item At the end of the file, add the following lines:

  \begin{lstlisting}
    personal-cipher-preferences AES256 TWOFISH AES192 AES
    personal-digest-preferences SHA512 SHA384 SHA256
    personal-compress-preferences ZLIB BZIP2 ZIP
    cert-digest-algo SHA256
  \end{lstlisting}
\end{enumerate}
\section*{Setup Keypair}

\begin{enumerate}
\item Open a command prompt and run the command \texttt{gpg -{}-gen-key}
\item When prompted, select option 1, ``RSA and RSA (default)''.
\item Use 4096 bits instead of the default.
\item Give an expiration date for the key (2y or 3y is a good choice).
  This can always be changed later, but it's good if you lose the key
  at any point in the future.  \textbf{Take note of when your key
    expires!  You can always change the expiration date later!}
\item Enter your name, enter your email address, but do not enter a
  comment.
\item Enter a passphrase and wait for the key to be generated.  This
  passphrase is the last line of defense if your private key is
  compromised, and you will need to enter it when you sign or decrypt
  messages.  \textbf{Do not forget this passphrase!}
\item Once your keypair has been generated, take note of the key ID
  that was printed to the terminal (it will look like \texttt{pub
    4096R/KEYID}).  Next, run the command \texttt{gpg -o revcert.asc
    -{}-gen-revoke \{KEYID\}}.  This will generate a revocation
  certificate, which allows you revoke this key if it ever gets
  stolen.  Don't give a reason (this is in case anything ever
  happens).  Keep it in a secure place (not, for instance, right in
  your home directory).
\item If you have more email addresses and names that you plan to use
  this key under, run the command \texttt{gpg -{}-edit-key \{KEYID\}}.
  At the prompt that appears, type \texttt{adduid}.  Again, enter your
  name and email (and no comment).  Do this for each email you want to
  add.  When you are done, select the primary user ID by running the
  commands \texttt{uid \{UID NUMBER\}} and \texttt{primary}.  Exit by
  typing \texttt{quit}, and be sure to save your changes.
\item When you are done setting up your key, run the command
  \texttt{gpg -{}-send-key \{KEYID\}} to submit your public key to the
  keyserver we set up.
\end{enumerate}

\section*{Signing, Verifying, Encrypting, and Decrypting Files}
You can now use the \texttt{gpg} command to sign, verify, encrypt, and
decrypt files:

\begin{lstlisting}
$ gpg -s filename             # Sign a file, makes filename.gpg
$ gpg --clearsign -a filename # Sign file in same file, makes filename.asc
$ gpg --verify filename.asc   # Verify a signature or clearsigned message
$ gpg -e filename             # Encrypt a file, makes filename.gpg
$ gpg -d filename.gpg         # Decrypt a file and print its contents
\end{lstlisting}

Additionally, if you navigate to any of the files in Windows Explorer
and right-click, you should see options to do all of the above.

\section*{Integration with Outlook, Thunderbird, etc.}
GPG4Win comes with an Outlook plugin to allow you to use GPG when
sending emails.  This should be installed automatically if you have
Outlook installed.  Thunderbird integrates with GPG very easily, but
there isn't enough space here to describe how.  Either Google it
(Thunderbird uses something called Enigmail) or ask one of us to help.

\section*{Signing Someone Else's Key}
When you sign someone else's key, you are saying that you have
verified that the information in the key is correct and that they own
the key.  To do this, you will need to see photo identification.
\textbf{Never, never sign a key if you haven't verified all of its
  information!}

Next, generally after the keysigning party, you do the actually
keysigning.  You need to make sure that the person owns all the email
addresses they say they do.  The way you verify this is a little
convoluted: for each User ID, you sign it separately, export it, and
send them encrypted to the email address you signed.  If the key has
three User IDs, and the person only controls two of them, only those
two emails can be encrypted, and so you will effectively only have
signed those two!

Follow these steps:

\begin{enumerate}
\item Once you have verified their name, download their public key
  running \texttt{gpg -{}-recv-keys \{THEIRKEYID\}}
\item Export the original key by running \texttt{gpg -{}-armor -o
    key.orig -{}-export \{THEIRKEYID\}}
\item For each User ID, do the following exactly:
  \begin{enumerate}
  \item Run \texttt{gpg -{}-edit-key \{THEIRKEYID\}}, and at the
    prompt, type \texttt{uid \{UIDNUMBERTOSIGN\}}.
  \item Type \texttt{sign} to sign the user ID, and then \texttt{quit}
    to exit this prompt.
  \item Export this key to a different file: \texttt{gpg -{}-armor -o
      key.uid\textit{n}signed -{}-export \{THEIRKEYID\}}.
  \item Encrypt this file, and send it by email to the email address
    you just signed.
  \item Delete the signed key (\texttt{gpg -{}-delete-keys
      \{THEIRKEYID\}}) and reimport the original (\texttt{gpg
      -{}-import key.orig}).  If you don't do this, the next exported
    key will have more than one User ID signed, which defeats the
    purpose of this!  Repeat for the remaining User IDs.
  \end{enumerate}
\item When the owner receives all the signatures, they will upload the
  new key to a keyserver, which you can then download.
\end{enumerate}

\section*{Having Your Own Key Signed}
If you are on the receiving end of the above process, be sure you
bring a photo ID (or more!) and your key information with you to the
keysigning.  Be sure to have written copies of your key id,
fingerprint, and all your user IDs to give to the keysigner, so they
can sign your key after the party.

Following the party, check your email addresses for encrypted messages
with the signatures.  Decrypt the messages, and import the attachments
using \texttt{gpg -{}-import filename}.  This will apply the
signatures to your key.  Then, send the newly signed key to a
keyserver so others can download it by running the command \texttt{gpg
  -{}-send-keys \{KEYID\}}.

\end{document}